\documentclass[landscape,a0paper,fontscale=0.292]{baposter}

\usepackage[vlined]{algorithm2e}
\usepackage{times}
\usepackage{calc}
\usepackage{url}
\usepackage{graphicx}
\usepackage{amsmath}
\usepackage{amssymb}
\usepackage{relsize}
\usepackage{multirow}
\usepackage{booktabs}
\usepackage{textcomp}
\usepackage{graphicx}
\usepackage{multicol}
\usepackage[T1]{fontenc}
\usepackage{ae}
%code chunk margins
\usepackage{listings}
\usepackage{caption}
\usepackage{subcaption}

\graphicspath{{images/}}

 %%%%%%%%%%%%%%%%%%%%%%%%%%%%%%%%%%%%%%%%%%%%%%%%%%%%%%%%%%%%%%%%%%%%%%%%%%%%%%%%
 %%%% Some math symbols used in the text
 %%%%%%%%%%%%%%%%%%%%%%%%%%%%%%%%%%%%%%%%%%%%%%%%%%%%%%%%%%%%%%%%%%%%%%%%%%%%%%%%
 % Format 
 \newcommand{\RotUP}[1]{\begin{sideways}#1\end{sideways}}


 %%%%%%%%%%%%%%%%%%%%%%%%%%%%%%%%%%%%%%%%%%%%%%%%%%%%%%%%%%%%%%%%%%%%%%%%%%%%%%%%
 % Multicol Settings
 %%%%%%%%%%%%%%%%%%%%%%%%%%%%%%%%%%%%%%%%%%%%%%%%%%%%%%%%%%%%%%%%%%%%%%%%%%%%%%%%
 \setlength{\columnsep}{0.7em}
 \setlength{\columnseprule}{0mm}


 %%%%%%%%%%%%%%%%%%%%%%%%%%%%%%%%%%%%%%%%%%%%%%%%%%%%%%%%%%%%%%%%%%%%%%%%%%%%%%%%
 % Save space in lists. Use this after the opening of the list
 %%%%%%%%%%%%%%%%%%%%%%%%%%%%%%%%%%%%%%%%%%%%%%%%%%%%%%%%%%%%%%%%%%%%%%%%%%%%%%%%
 \newcommand{\compresslist}{%
 \setlength{\itemsep}{1pt}%
 \setlength{\parskip}{0pt}%
 \setlength{\parsep}{0pt}%
 }


 %%%%%%%%%%%%%%%%%%%%%%%%%%%%%%%%%%%%%%%%%%%%%%%%%%%%%%%%%%%%%%%%%%%%%%%%%%%%%%
 % Formating
 \newcommand{\Matrix}[1]{\begin{bmatrix} #1 \end{bmatrix}}
 \newcommand{\Vector}[1]{\begin{pmatrix} #1 \end{pmatrix}}

 \newcommand*{\norm}[1]{\mathopen\| #1 \mathclose\|}% use instead of $\|x\|$
 \newcommand*{\abs}[1]{\mathopen| #1 \mathclose|}% use instead of $\|x\|$
 \newcommand*{\normLR}[1]{\left\| #1 \right\|}% use instead of $\|x\|$

 \newcommand*{\SET}[1]  {\ensuremath{\mathcal{#1}}}
 \newcommand*{\FUN}[1]  {\ensuremath{\mathcal{#1}}}
 \newcommand*{\MAT}[1]  {\ensuremath{\boldsymbol{#1}}}
 \newcommand*{\VEC}[1]  {\ensuremath{\boldsymbol{#1}}}
 \newcommand*{\CONST}[1]{\ensuremath{\mathit{#1}}}

 \DeclareMathOperator*{\argmax}{arg\,max}
 \DeclareMathOperator*{\diag}{diag}
 \DeclareMathOperator*{\argmin}{arg\,min}
 \DeclareMathOperator*{\vectorize}{vec}
 \DeclareMathOperator*{\reshape}{reshape}

 %-----------------------------------------------------------------------------
 % Differentiation
 \newcommand*{\Nabla}[1]{\nabla_{\!#1}}

 \renewcommand*{\d}{\mathrm{d}}
 \newcommand*{\dd}{\partial}

 \newcommand*{\At}[2]{\ensuremath{\left.#1\right|_{#2}}}
 \newcommand*{\AtZero}[1]{\At{#1}{\pp=\VEC 0}}

 \newcommand*{\diffp}[2]{\ensuremath{\frac{\dd #1}{\dd #2}}}
 \newcommand*{\diffpp}[3]{\ensuremath{\frac{\dd^2 #1}{\dd #2 \dd #3}}}
 \newcommand*{\diffppp}[4]{\ensuremath{\frac{\dd^3 #1}{\dd #2 \dd #3 \dd #4}}}
 \newcommand*{\difff}[2]{\ensuremath{\frac{\d #1}{\d #2}}}
 \newcommand*{\diffff}[3]{\ensuremath{\frac{\d^2 #1}{\d #2 \d #3}}}
 \newcommand*{\difffp}[3]{\ensuremath{\frac{\dd\d #1}{\d #2 \dd #3}}}
 \newcommand*{\difffpp}[4]{\ensuremath{\frac{\dd^2\d #1}{\d #2 \dd #3 \dd #4}}}

 \newcommand*{\diffpAtZero}[2]{\ensuremath{\AtZero{\diffp{#1}{#2}}}}
 \newcommand*{\diffppAtZero}[3]{\ensuremath{\AtZero{\diffpp{#1}{#2}{#3}}}}
 \newcommand*{\difffAt}[3]{\ensuremath{\At{\difff{#1}{#2}}{#3}}}
 \newcommand*{\difffAtZero}[2]{\ensuremath{\AtZero{\difff{#1}{#2}}}}
 \newcommand*{\difffpAtZero}[3]{\ensuremath{\AtZero{\difffp{#1}{#2}{#3}}}}
 \newcommand*{\difffppAtZero}[4]{\ensuremath{\AtZero{\difffpp{#1}{#2}{#3}{#4}}}}
 \renewcommand\labelenumi{(\theenumi)}

 %-----------------------------------------------------------------------------
 % Defined
 % How should the defined operator look like (:= or ^= ==)
 % (I want back my :=, it is so much better than ^= because (1) it has a
 % direction and (2) everyone here uses it.)
 %
 % Use :=
 %\newcommand*{\defined}{\ensuremath{\mathrel{\mathop{:}}=}}
 %\newcommand*{\definedRight}{\ensuremath{=\mathrel{\mathop{:}}}}
 % Use ^=
 \newcommand*{\defined}{\ensuremath{\triangleq}}
 \newcommand*{\definedRight}{\ensuremath{\triangleq}}
 % Use = with three bars
 %\newcommand*{\defined}{\ensuremath{?}}
 %\newcommand*{\definedRight}{\ensuremath{?}}

 %%%%%%%%%%%%%%%%%%%%%%%%%%%%%%%%%%%%%%%%%%%%%%%%%%%%%%%%%%%%%%%%%%%%%%%%%%%%%%
 % Symbols used in the paper

 %-----------------------------------------------------------------------------
 % The Methods
 \newcommand*{\ICIA}{\emph{ICIA}}
 \newcommand*{\CoDe}{\emph{CoDe}}
 \newcommand*{\LinCoDe}{\emph{LinCoDe}}
 \newcommand*{\CoNe}{\emph{CoNe}}
 \newcommand*{\CoLiNe}{\emph{CoLiNe}}
 \newcommand*{\LinCoLiNe}{\emph{LinCoLiNe}}

 % inter eye distance
 \newcommand*{\ied}{IED}

 %-----------------------------------------------------------------------------
 % Koerper
 %%\newcommand*{\RR}{\mathbb{R}}
 %\newcommand*{\RR}{{I\hspace{-3.5pt}R}}
 %\newcommand*{\RR}{{\mathrm{I\hspace{-2.7pt}R}}}

 \font\dsfnt=dsrom12

 \DeclareSymbolFont{nark}{U}{dsrom}{m}{n}
 \DeclareMathSymbol{\NN}{\dsfnt}{nark}{`N}
 \DeclareMathSymbol{\RR}{\dsfnt}{nark}{`R}
 \DeclareMathSymbol{\ZZ}{\dsfnt}{nark}{`Z}

 %-----------------------------------------------------------------------------
 % Domains
 \newcommand*{\D}{\mathcal{D}}
 \newcommand*{\I}{\mathcal{I}}

 %-----------------------------------------------------------------------------
 % Texture coordinates
 \newcommand*{\rr}{\VEC{r}}

 %-----------------------------------------------------------------------------
 % Parameters
 \newcommand*{\pt}{\VEC{\tau}}
 \newcommand*{\pr}{\VEC{\rho}}
 \newcommand*{\pp}{\VEC{p}}
 \newcommand*{\qq}{\VEC{q}}
 \newcommand*{\xx}{\VEC{x}}
 \newcommand*{\deltaq}{\Delta \qq}
 \newcommand*{\deltap}{\Delta \pp}
 \newcommand*{\zz}{\VEC{z}}
 \newcommand*{\pa}{\VEC{\alpha}}
 \newcommand*{\qa}{\VEC{\alpha}}
 \newcommand*{\pb}{\VEC{\beta}}

 %-----------------------------------------------------------------------------
 % Optimal appearance parameters
 \newcommand*{\pbh}[1]{\ensuremath{\hat{\pb}({#1})}}

 %-----------------------------------------------------------------------------
 % Warp basis
 \newcommand*{\M}[1]{\ensuremath{M({#1})}}
 \newcommand*{\LL}[1]{\ensuremath{L({#1})}}

 %-----------------------------------------------------------------------------
 % Matrices of the texture model
 \newcommand*{\AM}[1]{\ensuremath{\Lambda(#1)}}               % Lambda(beta) 
 \newcommand*{\AMr}[2]{\ensuremath{\Lambda(#1; #2)}}        % Lambda(r, beta)

 \newcommand*{\As}{A}         % Continuous Basis symbol
 \newcommand*{\afs}{a}        % Continuous mean symbol
 \newcommand*{\A}[1]{\As(#1)}         % Continuous Basis
 \newcommand*{\af}[1]{\afs(#1)}        % Continuous mean


 %-----------------------------------------------------------------------------
 % Matrices of the shape model
 \newcommand*{\MU}{\VEC{\mu}}
 \newcommand*{\MM}{\MAT{M}}

 %-----------------------------------------------------------------------------
 %% The project out matrix and operator
 \newcommand*{\INT}{\MAT{P}}
 \newcommand*{\INTf}{P}

 %-----------------------------------------------------------------------------
 % The identity matrix
 \newcommand*{\EYEtwo}{\Matrix{1 & 0\\0&1}}
 \newcommand*{\EYE}{\MAT E}
 \newcommand*{\EYEf}{E}

 % Wether to use subscripts or brackets for some function arguments
 % can be decided by commenting out the corresponding functions underneath
 %-----------------------------------------------------------------------------
 % Mapping
 \newcommand*{\Cs}[1]{\ensuremath{C^{#1}}} % C symbol
 \newcommand*{\C}[2]{\ensuremath{C^{#1}(#2)}} % Use C with brackets

 %-----------------------------------------------------------------------------
 % Objective function
 \newcommand*{\Fs}{\ensuremath{F}}              % F symbol
 \newcommand*{\F}[1]{\ensuremath{\Fs(#1)}}       % Use F with brackets    F(q)

 %-----------------------------------------------------------------------------
 % Approximated objective functions
 \newcommand*{\FFs}{\tilde{F}}                     % ~F symbol
 \newcommand*{\FF}[1]{\ensuremath{\FFs(#1)}}       % Use ~F with brackets    F(q)

 %-----------------------------------------------------------------------------
 % residual function
 \newcommand*{\es}{\ensuremath{f}}              % R symbol

 \newcommand*{\e}[1]{\ensuremath{\es(#1)}}         % R(q)
 \newcommand*{\er}[2]{\ensuremath{\es(#1; #2)}}    % R(r; q)

 %-----------------------------------------------------------------------------
 % Approximated residual functions
 \newcommand*{\ees}{\tilde{f}}                       % ~R symbol
 \newcommand*{\ee}[1]{\ensuremath{\ees(#1)}}       % ~R(q)
 \newcommand*{\eer}[2]{\ensuremath{\ees(#2; #1)}}  % ~R(r; q)

 %-----------------------------------------------------------------------------
 % Warps
 \newcommand*{\Vs}{\ensuremath{V}}
 \newcommand*{\VLins}{\ensuremath{\Vs^{\text{Ortho}}}}
 \newcommand{\VModels}{\ensuremath{\Vs^{\text{Model}}}}
 \newcommand*{\Ws}{\ensuremath{W}}

 \newcommand{\V}[1]{\ensuremath{\Vs(#1)}}
 \newcommand{\VModel}[1]{\ensuremath{\VModels(#1)}}
 \newcommand{\Vr}[2]{\ensuremath{\Vs(#1; #2)}}
 \newcommand{\VInvr}[2]{\ensuremath{\Vs^{-1}(#1; #2)}}
 \newcommand{\VrLin}[2]{\ensuremath{\VLins(#1; #2)}}
 \newcommand{\W}[1]{\ensuremath{\Ws(#1)}}
 \newcommand{\Winv}[1]{\ensuremath{\Ws^{-1}(#1)}}
 \newcommand{\Wr}[2]{\ensuremath{\Ws(#1; #2)}}


%%%%%%%%%%%%%%%%%%%%%%%%%%%%%%%%%%%%%%%%%%%%%%%%%%%%%%%%%%%%%%%%%%%%%%%%%%%%%
%% Begin of Document
%%%%%%%%%%%%%%%%%%%%%%%%%%%%%%%%%%%%%%%%%%%%%%%%%%%%%%%%%%%%%%%%%%%%%%%%%%%%%
\begin{document}
%%%%%%%%%%%%%%%%%%%%%%%%%%%%%%%%%%%%%%%%%%%%%%%%%%%%%%%%%%%%%%%%%%%%%%%%%%%%%
%% Here starts the poster
%%---------------------------------------------------------------------------
%% Format it to your taste with the options
%%%%%%%%%%%%%%%%%%%%%%%%%%%%%%%%%%%%%%%%%%%%%%%%%%%%%%%%%%%%%%%%%%%%%%%%%%%%%
\begin{poster}{
 % Show grid to help with alignment
 grid=false,
 % Column spacing
 colspacing=0.7em,
 % Color style
 headerColorOne=black,
 borderColor=black,
 % Format of textbox
 textborder=faded,
 % Format of text header
 headerborder=open,
 headershape=roundedright,
 headershade=plain,
 background=none,
 bgColorOne=cyan!10!white,
 headerFontColor=white,
 textborder=roundedleft,
 boxshade=none,
 headerheight=0.12\textheight}
 % Eye Catcher
 {
      \includegraphics[width=0.20\linewidth]{UCBerkeley_wordmark_black.png}
 }
 % Title
 {\sc\huge Evaluation of Solvers on Acoustic Waves LF-Domain Modeling}
 % Authors
 {Hussain AlSalem\\[1em]
 {\texttt{geohussain@berkeley.edu}}}
 % University logo
 {
  \begin{tabular}{r}
    \includegraphics[height=0.10\textheight]{ucberkeleyseal_line_k.png}
  \end{tabular}
 }

%%%%%%%%%%%%%%%%%%%%%%%%%%%%%%%%%%%%%%%%%%%%%%%%%%%%%%%%%%%%%%%%%%%%%%%%%%%%%%
%%% Now define the boxes that make up the poster
%%%---------------------------------------------------------------------------
%%% Each box has a name and can be placed absolutely or relatively.
%%% The only inconvenience is that you can only specify a relative position 
%%% towards an already declared box. So if you have a box attached to the 
%%% bottom, one to the top and a third one which should be inbetween, you 
%%% have to specify the top and bottom boxes before you specify the middle 
%%% box.
%%%%%%%%%%%%%%%%%%%%%%%%%%%%%%%%%%%%%%%%%%%%%%%%%%%%%%%%%%%%%%%%%%%%%%%%%%%%%%

%%%%%%%%%%%%%%%%%%%%%%%%%%%%%%%%%%%%%%%%%%%%%%%%%%%%%%%%%%%%%%%%%%%%%%%%%%%%%%
  \headerbox{Introduction}{name=Introduction,column=0,row=0,span=2}{
%%%%%%%%%%%%%%%%%%%%%%%%%%%%%%%%%%%%%%%%%%%%%%%%%%%%%%%%%%%%%%%%%%%%%%%%%%%%%%
  \textbf{Challenge:} Current IDR (iterative Krylov subspace) solver can be slower than Direct SuperLU for multi-RHS when forward modelling acoustic waves in Laplace-Fourier (LF) domain for 2nd and 4th order Finite-Difference (FD) discretization schemes.
   \\\\
   \textbf{Objective:} Find a compromise of when direct SuperLU Solver is better than IDR and vice-versa:
   \begin{enumerate}
   \item Strong and weak scaling comparisons between the two solvers. \label{item:1}
   \item Maximum problem size that SuperLU can handle on fully packed nodes (default memory per core). \label{item:2}
   \item Advantages of using unpacked nodes (large memory per core) on Edison for SuperLU. \label{item:3}
   \end{enumerate}
  }
%%%%%%%%%%%%%%%%%%%%%%%%%%%%%%%%%%%%%%%%%%%%%%%%%%%%%%%%%%%%%%%%%%%%%%%%%%%%%%
  \headerbox{Governing Equations}{name=abstract,column=0,below=Introduction}{
%%%%%%%%%%%%%%%%%%%%%%%%%%%%%%%%%%%%%%%%%%%%%%%%%%%%%%%%%%%%%%%%%%%%%%%%%%%%%%
	The first-order system can be recast as a second-order equation in pressure after elimination of the particle velocities in the frequency-domain 3D acoustic first-order velocity-pressure system, that leads to a generalization of the Helmholtz equation:
	\begin{equation}\label{eq:helm_eq}
		\begin{split}
			\frac{\omega^2}{\kappa(\mathbf{x})}p(\mathbf{x},\omega)+\frac{\partial}{\partial x}b(\mathbf{x})\frac{\partial p(\mathbf{x},\omega)}{\partial x} \\
			+ \frac{\partial}{\partial y}b(\mathbf{x})\frac{\partial p(\mathbf{x},\omega)}{\partial y}+\frac{\partial}{\partial z}b(\mathbf{x})\frac{\partial p(\mathbf{x},\omega)}{\partial z} \\
			= s(\mathbf{x},\omega).
		\end{split}
	\end{equation}
	Equation~\ref{eq:helm_eq} can be recast in matrix form:
	\begin{equation}\label{eq:matrix_form}
		\begin{split}
			[\mathbf{M}+\mathbf{S}]\mathbf{u}=\mathbf{Au}=\mathbf{b}.	
		\end{split}
	\end{equation}
}
 %%%%%%%%%%%%%%%%%%%%%%%%%%%%%%%%%%%%%%%%%%%%%%%%%%%%%%%%%%%%%%%%%%%%%%%%%%%%%%
   \headerbox{Parallel Results on Fully Packed Nodes}{name=speed,column=2,row=0,span=2}{
 %%%%%%%%%%%%%%%%%%%%%%%%%%%%%%%%%%%%%%%%%%%%%%%%%%%%%%%%%%%%%%%%%%%%%%%%%%%%%%
   \begin{multicols}{2}
   {
     \includegraphics[width=\linewidth]{edison_compare_strong.png}
   }
   {
     \includegraphics[width=\linewidth]{edison_ratio.png}
   }
   {
     \includegraphics[width=\linewidth]{edison_speedup.png}
   }
   {
     \includegraphics[width=\linewidth]{edison_weak_2nd.png}
   }
   {
     \includegraphics[width=\linewidth]{edison_weak_4th.png}
   }
   \end{multicols}
   }
 %%%%%%%%%%%%%%%%%%%%%%%%%%%%%%%%%%%%%%%%%%%%%%%%%%%%%%%%%%%%%%%%%%%%%%%%%%%%%%
   \headerbox{References}{name=references,column=0,above=bottom}{
 %%%%%%%%%%%%%%%%%%%%%%%%%%%%%%%%%%%%%%%%%%%%%%%%%%%%%%%%%%%%%%%%%%%%%%%%%%%%%%
     \smaller
     
     \bibliographystyle{ieee}
     \renewcommand{\section}[2]{\vskip 0.05em}
       \begin{thebibliography}{1}\itemsep=-0.01em
       \setlength{\baselineskip}{0.4em}

       \bibitem{amberg07:nicp}
       Li, Xiaoye S and Demmel, et. al
       \newblock On SuperLU user\textquotesingle s guide
       \newblock In 2011.
	   
	   \bibitem{sonneveld2008idr:nicp}
	   Sonneveld, Peter and van Gijzen, Martin B
	   \newblock On IDR (s): A family of simple and fast algorithms for solving large nonsymmetric systems of linear equations
	   \newblock In 2008.

       \end{thebibliography}
   }

 %%%%%%%%%%%%%%%%%%%%%%%%%%%%%%%%%%%%%%%%%%%%%%%%%%%%%%%%%%%%%%%%%%%%%%%%%%%%%%
   \headerbox{2nd vs 4th Order Schemes}{name=serial,column=0,above=references,below=abstract}{
 %%%%%%%%%%%%%%%%%%%%%%%%%%%%%%%%%%%%%%%%%%%%%%%%%%%%%%%%%%%%%%%%%%%%%%%%%%%%%%
   \textbf{A} matrix for both second and fourth order schemes:
   {
     \includegraphics[width=0.45\linewidth]{A_2nd.png}
   }
   {
     \includegraphics[width=0.45\linewidth]{A_4th.png}
   }
   }

 %%%%%%%%%%%%%%%%%%%%%%%%%%%%%%%%%%%%%%%%%%%%%%%%%%%%%%%%%%%%%%%%%%%%%%%%%%%%%%

   \headerbox{UnPacked Nodes Results}{name=parallel_tests,column=2,span=1,below=speed,above=bottom}{
 %%%%%%%%%%%%%%%%%%%%%%%%%%%%%%%%%%%%%%%%%%%%%%%%%%%%%%%%%%%%%%%%%%%%%%%%%%%%%%
   {
     \includegraphics[width=0.95\linewidth]{Unpacked_edison_2nd_ratio.png}
   }
   }
 %%%%%%%%%%%%%%%%%%%%%%%%%%%%%%%%%%%%%%%%%%%%%%%%%%%%%%%%%%%%%%%%%%%%%%%%%%%%%%

   \headerbox{Observations}{name=parallel_tests,column=3,span=1,below=speed,above=bottom}{
 %%%%%%%%%%%%%%%%%%%%%%%%%%%%%%%%%%%%%%%%%%%%%%%%%%%%%%%%%%%%%%%%%%%%%%%%%%%%%%
   \begin{enumerate}
	   \item For medium-sized problems and using fully packed nodes (2.67GB per core), IDR(s) is 10 times faster than SuperLU. 
	   \item For large-sized problems and using unpacked nodes (32GB per core), IDR(s) is 100 times faster than SuperLU.
	   \item SuperLU runs out of memory for medium-sized problems with fourth order discretization. Imagine the maximum size it can handle for elastic problems!
	   \item SuperLU is more efficient in strong-scaling.
   \end{enumerate}
   }
 %%%%%%%%%%%%%%%%%%%%%%%%%%%%%%%%%%%%%%%%%%%%%%%%%%%%%%%%%%%%%%%%%%%%%%%%%%%%%%
   \headerbox{SuperLU Pros \& Cons}{name=Parallelization,column=1,span=1,above=bottom}{
 %%%%%%%%%%%%%%%%%%%%%%%%%%%%%%%%%%%%%%%%%%%%%%%%%%%%%%%%%%%%%%%%%%%%%%%%%%%%%%
   \textbf{Advantages} of using SuperLU:
   \begin{enumerate}
   \item Can solve for multiple RHS (sources) cheaply once matrix \textbf{A} is factorized.
   \item Scales better when using multiple processors compared to the current IDR(s) method.
   \end{enumerate}
   \textbf{Drawbacks} of using SuperLU:
   \begin{enumerate}
   \item Out of memory (OOM) errors during fill-in for large matrices.
   \item SuperLU is generally slower than iterative methods when solving for one RHS.
   \end{enumerate}
   }

 %%%%%%%%%%%%%%%%%%%%%%%%%%%%%%%%%%%%%%%%%%%%%%%%%%%%%%%%%%%%%%%%%%%%%%%%%%%%%%%
  \headerbox{Parallel SuperLU Algorithm}{name=algorithm,column=1,above=Parallelization,below=Introduction}{
 %%%%%%%%%%%%%%%%%%%%%%%%%%%%%%%%%%%%%%%%%%%%%%%%%%%%%%%%%%%%%%%%%%%%%%%%%%%%%%%
  \begin{algorithm}[H]
    \dontprintsemicolon
    \linesnumbered
    \For{block $K=1$ $\mathbf{to}$ $N$}{
      \nl Factorize $A(K:N,K) \rightarrow L(K:N,K)$\\
	  (may involve pivoting)\\
	  \nl Compute $U(K,K+1:N)$\\
	  (via a sequence of triangular solves)\\
	  \nl Update $A(K+1:N,K+1:N) \leftarrow$ \\
	             $A(K+1:N,K+1:N)-L(K+1:N,K) \cdot U(K,K+1:N)$ \\
	  (via a sequence of calls to GEMM)
      % \Repeat{converged}{
      %   \nl Calculate $\Nabla{\pp}{\tilde{F}(\qq,\VEC 0)}$, $F(\qq)$\;
      %   \eIf{$F(\qq) < F(\qq_{\text{best}})$}{
      %     %\nl Calculate distance between best warp estimate and current warp estimate to test for convergence\;
      %     \nl $\qq_{\text{best}} \gets \qq$\;
      %     %\If{More than three successiv updates}{ (Too much detail)
      %       \nl Increase $\kappa$\;
      %     %}
      %   }{
      %     \If{$\kappa$ smaller than threshold}{
      %       \nl return\;
      %     }%{
      %        decrease $\kappa$\;
      %     %}
      %   }
      %   \nl Calculate $\pp$ from $\Nabla{\pp}{\tilde{F}(\qq_{best},\pp)}$ and $\kappa$\;
      %   \nl $\qq \gets \C{\circ}{\qq, \pp}$
      % }
    }
  \end{algorithm}
  The algorithm is chosen for parallel SuperLU:
  \begin{enumerate}
	  \item The sparsity structure can be determined before numerical factorization because of static pivoting.
	  \item All the GEMM (general matrix multiplication) updates to the trailing submatrix are independent and so can be done in parallel.
	  
   \end{enumerate}
  
   }
\end{poster}%
%
\end{document}
